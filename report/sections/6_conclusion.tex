\section{Conclusions}
The task that was analyzed is the music genre classification.
Various methods were tried.
First an approach based on handcrafted features was tried, obtaining accuracy 0.495.
Then, considering the state of the art, CNNs with mel-spectogram as input were chosen. 
Initially the CNNs were approached from scratch using an architecture suggested by \cite{kostrzewa2021music} and the accuracy was 0.3975. 
Then hyperparameters optimization and data augmentation were made. 
This path was found to be the worst, obtaining 0.3900 without data augmentation and 0.3787 with it.
Finally, considering the quantity of data available and the results previously obtained, transfer learning were tried.
Two Top-5 accuracy CNN on imagenet were used to do feature extractions.
The extracted features were reduced in size with PCA, which has been shown to improve the results.
In the end the best model obtained was the one using ResNet50 at cut \texttt{conv5\_block1\_2\_relu} with radial SVM, which has fewer parameters than VGG16 and is faster, with accuracy of 0.5012, reaching results similar to the paper previously cited 0.5163.
The use of spectograms with CNNs was proved to be the best method in the literature to deal with this task.
In order to obtain better results, however, a greater amount of data is required.
