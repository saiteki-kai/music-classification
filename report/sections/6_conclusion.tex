\section{Conclusions}
The task that was analyzed is the music genre classification.
Various methods were tried.
First an approach based on handcrafted features was tried, obtaining accuracy 0.495, then
considering the state of the art, then CNN with mel-spectogram as input were chosen,
initially from scratch using an architecture suggested by \cite{kostrzewa2021music} and obtaining accuracy 0.3975
to then do hyperparameters optimization and data augmentation. This path was found to be the worst, obtaining 0.3900
for without data augmentation and 0.3787 with it.
Finally, transfer learning were tried considering the quantity of data available and the results previously obtained.
Two Top-5 accuracy CNN on imagenet were used to do feature extractions.
The extracted features were reduced in size with pca which has been shown to improve the results.
In the end the best model obtained was the one with ResNet50 at cut \texttt{conv5\_block1\_2\_relu} with radial SVM which has fewer parameters than VGG16 and is faster
with accuracy 0.5012, reaching results similar to the paper previously cited.
The use of spectograms with CNNs was proved to be the best method in the literature to deal with this task.
In order to obtain better results, however, a greater amount of data is required.
