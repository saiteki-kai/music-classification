%%%%%%%%%%%%%%%%%%%%%%%%%%%%%%%%%%%%%%%%%
% University Assignment Title Page 
% LaTeX Template
% Version 1.0 (27/12/12)
%
% This template has been downloaded from:
% http://www.LaTeXTemplates.com
%
% Original author:
% WikiBooks (http://en.wikibooks.org/wiki/LaTeX/Title_Creation)
%
% License:
% CC BY-NC-SA 3.0 (http://creativecommons.org/licenses/by-nc-sa/3.0/)
%
%%%%%%%%%%%%%%%%%%%%%%%%%%%%%%%%%%%%%%%%%

%----------------------------------------------------------------------------------------
%	PACKAGES AND OTHER DOCUMENT CONFIGURATIONS
%----------------------------------------------------------------------------------------

\documentclass[12pt]{article}
\usepackage[english]{babel}
\usepackage[utf8x]{inputenc}
\usepackage{amsmath}
\usepackage{graphicx}
\usepackage[hidelinks]{hyperref}
\usepackage{multirow}
\usepackage{subcaption}

\begin{document}

%----------------------------------------------------------------------------------------
%	TITLEPAGE
%----------------------------------------------------------------------------------------

\begin{titlepage}

    \newcommand{\HRule}{\rule{\linewidth}{0.5mm}} % Defines a new command for the horizontal lines, change thickness here

    \center % Center everything on the page

    %----------------------------------------------------------------------------------------
    %	HEADING SECTIONS
    %----------------------------------------------------------------------------------------

    \textsc{\LARGE Università degli studi di Milano-Bicocca}\\[1cm] % Name of your university/college
    \textsc{\Large Advanced Machine Learning }\\[0.3cm] % Major heading such as course name
    \textsc{\large Final Project}\\[0.1cm] % Minor heading such as course title

    %----------------------------------------------------------------------------------------
    %	TITLE SECTION
    %----------------------------------------------------------------------------------------

    \HRule \\[0.4cm]
    { \huge \bfseries Music Genre Classification}\\[0.4cm] % Title of your document
    \HRule \\[1.5cm]

    %----------------------------------------------------------------------------------------
    %	AUTHOR SECTION
    %----------------------------------------------------------------------------------------

    \large
    \emph{Authors:}\\
    Davide Pietrasanta - 844824 - d.pietrasanta@campus.unimib.it  \\   % Your name
    Giuseppe Magazzù - 829612 - g.magazzu1@campus.unimib.it   \\ % Your name
    Gaetano Magazzù - 829685 - g.magazzu2@campus.unimib.it   \\[0.4cm] % Your name

    %----------------------------------------------------------------------------------------
    %	DATE SECTION
    %----------------------------------------------------------------------------------------

    {\large \today}\\[2cm] % Date, change the \today to a set date if you want to be precise

    %----------------------------------------------------------------------------------------
    %	LOGO SECTION
    %----------------------------------------------------------------------------------------

    \includegraphics{images/logo.png}\\[1cm] % Include a department/university logo - this will require the graphicx package

    %----------------------------------------------------------------------------------------

    \vfill % Fill the rest of the page with whitespace

\end{titlepage}



%----------------------------------------------------------------------------------------
%	ABSTRACT
%----------------------------------------------------------------------------------------

\begin{abstract}
The problem faced is that of the classification of music 
genres from audio files.
Various approaches were used.
Starting from approaches based on handcrafted features, the project moved on to approaches based on the mel-spectogram and CNNs. 
An attempt was therefore made to transfer learning methods and data augmentation to achieve better performance.
The best results were achieved by using ResNet50 with \textbf{0.5012} of accuracy on the test set.

https://github.com/saiteki-kai/music-classification
\end{abstract}

%----------------------------------------------------------------------------------------
%	SECTIONS
%----------------------------------------------------------------------------------------

\section{Introduction}
Nowadays, with the high traffic of multimedia data, it becomes more and more 
necessary to be able to organize, in an automatic way, one's music. 
With a better audio classification we can improve recommendation algorithms, and discover trends and listener preferences through data analysis. \\

The task we are considering for this project is the classification of music 
genres from audio files.
It is assumed that only one musical genre is associated with each audio track. \\

We've approached the problem with an initial study of the theory and the state of the art. 
We then tried to overcome the results obtained with approaches seen in class, as Transfer Learning. \\

You can look at the code at the following link: \\
{https://github.com/saiteki-kai/music-classification}
\section{Datasets}
% In this section the available data sets must be presented.
% The term dataset refers to any type of information source, for example web services for geolocation fall into this category.
% In addition, all necessary data manipulation processes, such as cleaning and enrichment with external sources, must be presented and discussed.

The dataset chosen is the Free Music Archive (FMA).
The dataset contains 106574 high quality audio tracks lasting approximately 30 seconds. All the tracks are common creative licensed.

Each track is associated with additional information about the artist (name, location, bio, etc.), the album (title, listens, comments, etc.) and the track itself (title, creation date, duration, genres, etc.). 

The genres are organized in a hierarchy of 161 unbalanced classes of different genres.

%Pre-calculated features are also provided: STFT Chromagram, CQT Chromagram, Chroma Energy Normalized (CENS), Tonal Centroid Features (Tonnetz),
%RMSE, Zero Crossing Rate, Spectral Centroid, Spectral Bandwidth, Spectral Contrast, Spectral Rolloff, Mel Frequency Cepstral Coefficients (MFCC). 
%Statistical moments are provided for each of these features: mean, std, skew, kurtosis, median, min, max.
Pre-computed features are also provided such as the statistical moments of some spectral and temporal features.

The dataset propose a train/validation/test (80\%/10\%/10\%) split and three subsets (small, medium, large).
In this work only the small one was used which consists of 8000 tracks and 8 balanced genres.
% \begin{itemize}
%   \item Small: 8000 tracks, 8 balanced genres 
%   \item Medium: 25000 tracks, 16 unbalanced genres
%   \item Large: 106574 tracks, 161 unbalanced genres
% \end{itemize}

\subsection{Features}
% TODO: introduzione al metodo con CNN?

Analyzing the raw audio tracks it was found that the sampling frequency varies between 22050Hz and 48000Hz. 
Most of the tracks have a sampling rate of 44100Hz.
It was decided to resample all tracks to 22050Hz.

Furthermore, it has been found that the durations of the audio tracks are not all 30 seconds long.
The range of duration values that was found was from 0 to 30.02 seconds.
Then the length was reduced for all tracks to 29.97 seconds and the shorter length samples were removed due to incorrect length metadata.

% TODO: mettere il link? https://github.com/mdeff/fma/wiki#excerpts-shorter-than-30s-and-erroneous-audio-length-metadata

The Mel spectrogram was calculated on the raw audio with the following parameters: 
\begin{itemize}
  \item Sample rate: 22050Hz
  \item Window Size: 2048
  \item Hop Length: 512
  \item Mel bins: 128
\end{itemize}


\section{The Methodological Approach}

This is the central and most important section of the report.
Its objective must be to show, with linearity and clarity, the steps that have led to the definition of a decision model.
The description of the working hypotheses, confirmed or denied, can be found in this section together with the description of the subsequent refining processes of the models.
Comparisons between different models (e.g. heuristics vs. optimal models) in terms of quality of solutions, their explainability and execution times are welcome.

Do not attempt to describe all the code in the system, and do not include large pieces of code in this section, use pseudo-code where necessary.
Complete source code should be provided separately (in Appendixes, as separated material or as a link to an on-line repo).
Instead pick out and describe just the pieces of code which, for example:
\begin{itemize}
    \item are especially critical to the operation of the system;
    \item you feel might be of particular interest to the reader for some reason;
    \item illustrate a non-standard or innovative way of implementing an algorithm, data structure, etc..
\end{itemize}

You should also mention any unforeseen problems you encountered when implementing the
system and how and to what extent you overcame them. Common problems are:
difficulties involving existing software.

\section{Results and Evaluation}
The Results section is dedicated to presenting the actual results (i.e. measured and calculated quantities), not to discussing their meaning or interpretation.
The results should be summarized using appropriate Tables and Figures (graphs or schematics).
Every Figure and Table should have a legend that describes concisely what is contained or shown.
Figure legends go below the figure, table legends above the table.
Throughout the report, but especially in this section, pay attention to reporting numbers with an appropriate number of significant figures.



\begin{table}[ht]
\begin{tabular}{|l|l|l|l|l|}
\hline
                    & Test accuracy & Test F1 & Test Loss \\ \hline
CNN                 & 0.3975        & 0.2414  & 3.7596    \\ \hline
CNN augmented       &                     &               & \\ \hline
Tuned CNN           & 0.3900        & 0.2414  & 2.3860\\ \hline
Tuned CNN augmented &                     &               & \\ \hline
\end{tabular}
\end{table}

\begin{table}[ht]
\begin{tabular}{|l|l|l|l|}
\hline
                    & Train accuracy & Train F1 & Train Loss \\ \hline
CNN                 & 0.999          & 0.2414   & 0.0024    \\ \hline
CNN augmented       &                     &               & \\ \hline
Tuned CNN           & 0.9998         & 0.2414   & 0.0065    \\ \hline
Tuned CNN augmented &                     &               & \\ \hline
\end{tabular}
\end{table}

\begin{table}[ht]
\begin{tabular}{|l|l|l|l|}
\hline
                    & Validation accuracy & Validation F1 & Validation Loss \\ \hline
CNN                & 0.4800              & 0.2414        & 3.0486          \\ \hline
CNN augmented       &                     &               &                 \\ \hline
Tuned CNN           & 0.4175              & 0.2414        & 2.0767          \\ \hline
Tuned CNN augmented &                     &               &                 \\ \hline
\end{tabular}
\end{table}

\begin{figure}[ht]
\centering
\includegraphics[scale=0.6]{images/2021-val-train.png}
\caption{CNN's Accuracy and Loss function of the training and validation set.}
\label{fig:Acc_Loss_2021}
\end{figure}


\begin{figure}[ht]
\centering
\includegraphics[scale=0.6]{images/tuned-val-train.png}
\caption{Tuned CNN's Accuracy and Loss function of the training and validation set.}
\label{fig:Acc_Loss_tuned}
\end{figure}
\section{Discussion}
With the handcrafted features, accuracy results on the test set are equal to \textbf{0.495}. 
The prediction time is fast as the number of parameters is low compared to CNNs.

Regarding the approach based on spectrograms and Convolutional Neural Networks it is noted how the models predict more or less with the same speed. It is also noticeable that they are all overfitting.
The optimization of the hyper-parameters also brings greater stability to the validation curve.
It should also be noted that the data augmentation, in this case, leads the task to be more difficult, thus leading to worse results.
The best performing CNN model is the one described in the paper by "Daniel Kostrzewa" \cite{kostrzewa2021music}, with an accuracy of \textbf{0.3975}. It is also important to say that the performances obtained are lower than those stated in the paper, which reaches \textbf{0.5163} of accuracy. This could be due to a different pre-processing or a number higher than 50 epochs.

With the approach based on Transfer Learning we have obtained the best results. Using ResNet50 leads to \textbf{0.5012} of accuracy.

From the table [\ref{tab:my-table}] it was noted how decreasing the percentage of explained variance allows to obtain better results with a greater reduction.

From table [\ref{tab:my-table2}] it possible to see how the linear SVM gets worse going to lower layers. Probably because the data becomes less linearly separable.
The radial SVM instead gets improvements at lower levels, while for the MLP its trend is not clear, probably due to the scarcity of data.

From table [\ref{tab:my-table3}] it is also possible to see here the same effect on the linear SVM as on the radial SVM, while the MLP also improves as you go down in depth.

In the end, the two best models obtained are VGG16 at the \texttt{block5\_pool} level with radial SVM and with an accuracy of 0.4968, while ResNet50 at the \texttt{conv5\_block1\_2\_relu} level with radial SVM with an accuracy of 0.5012.

\noindent
Comparing the models we can see that the two confusion matrices are very similar as expected from their very similar accuracy.
What differentiates the two models are the number of parameters 138,357,544 and 25,636,712 and the prediction times of 75.913ms/26.601ms and 54.615ms/27.021ms for respectivly CPU/GPU.

\vspace{4mm}
\noindent
Possible ways to improve could be:
\begin{itemize}
  \item Better data augmentation (for example by acting directly on the temporal domain)
  \item Other subset of the dataset (medium, large)
  \item Optimization of hyper-parameters from start to finish
  \item Consider a different loss function and consider a greater number of epochs
  \item Other features as MFCCs and its derivatives
  \item Other models such as LSTM or RCNN could also be tried, even if the state of the art shows that they are not as performing as CNNs \cite{kostrzewa2021music}.
\end{itemize}

\section{Conclusions}

Various methods for classifying audio genres have been tried.
The approach based on spectrograms and Convolutional Neural Networks was the worst. 
The Handcrafted Features based approach proved to be a good alternative. 
The approach with the best results turned out to be the one based on Transfer Learning.
The use of spectrograms is however indispensable for these types of problems.


%----------------------------------------------------------------------------------------
%	APPENDIX
%----------------------------------------------------------------------------------------

\appendix
\section{Appendix}

\subsection{Hyperband}
    \cite{li2016novel}



%----------------------------------------------------------------------------------------
%	BIBLIOGRAPHY
%----------------------------------------------------------------------------------------

\bibliographystyle{IEEEtran}
\bibliography{references.bib}

\end{document}
