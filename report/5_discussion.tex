\section{Discussion}

With the Handcrafted Features, accuracy results on the test set are equal to \textbf{0.495}.

Regarding the approach based on spectrograms and Convolutional Neural Networks it is noted how the models predict more or less with the same speed. It is also noticeable that they are all overfitting.
The optimization of the hyper-parameters also brings greater stability to the validation curve.
It should also be noted that the data augmentation, in this case, leads the task to be more difficult, thus leading to worse results.
The best performing CNN model is the one described in the paper by "Daniel Kostrzewa" \cite{kostrzewa2021music}, with an accuracy of \textbf{0.3975}. It is also important to say that the performances obtained are lower than those stated in the paper, which reaches \textbf{0.5163} of accuracy. This could be due to a different pre-processing or a number higher than 50 epochs.

With the approach based on Transfer Learning we have obtained the best results. Using XX leads to \textbf{YY} of accuracy.

Possible ways to improve could be: better data augmentation, optimization of hyper-parameters from start to finish, consider a different loss function and consider a greater number of epochs.

Other models such as LSTM or RCNN could also be tried, even if the state of the art shows that they are not as performing as CNNs \cite{kostrzewa2021music}.






